\documentclass{article}
\usepackage[utf8]{inputenc}
\usepackage{listings}
\usepackage{xcolor}

% --- Definicja kolorów ---
\definecolor{codegreen}{rgb}{0,0.6,0}
\definecolor{codegray}{rgb}{0.5,0.5,0.5}
\definecolor{codepurple}{rgb}{0.58,0,0.82}
\definecolor{backcolour}{rgb}{0.95,0.95,0.92}

% --- Definicja języka PDDL ---
\lstdefinelanguage{PDDL}{
  sensitive=false,    % PDDL nie jest wrażliwy na wielkość liter
  morecomment=[l]{;}, % Komentarze w PDDL zaczynają się od średnika
  alsoletter={-,:},   % Zezwala na myślniki i dwukropki w słowach kluczowych
  morekeywords={
    define, domain, problem, and, or, not, imply, exists, forall, when,
    increase, decrease, assign, scale-up, scale-down,
    :domain, :requirements, :types, :constants, :predicates, :functions,
    :action, :parameters, :precondition, :effect, 
    :durative-action, :duration, :condition, 
    :objects, :init, :goal, :metric
  }
}

% --- Konfiguracja ogólnego stylu ---
\lstset{
  language=PDDL,
  backgroundcolor=\color{backcolour},   
  commentstyle=\color{codegreen}\itshape,
  keywordstyle=\color{blue}\bfseries,
  numberstyle=\tiny\color{codegray},
  stringstyle=\color{codepurple},
  basicstyle=\ttfamily\footnotesize,
  breakatwhitespace=false,         
  breaklines=true,                 
  captionpos=b,                    
  keepspaces=true,                 
  numbers=left,                    
  numbersep=5pt,                  
  showspaces=false,                
  showstringspaces=false,
  showtabs=false,                  
  tabsize=2
}

\usepackage{graphicx}


\title{Classic planning with PDDL}
\author{Aleksandra Słomska \\ Zofia Narloch}


\begin{document}

\maketitle

\section{Exercise 1.1: Emergency Service Logistics, Initial Release}

\begin{lstlisting}[caption={Outcome of the first problem - one person and one box}, label={lst:pddl-example}]
( PICK-UP-CRATE CRATE1 DEPOT DRONE1 RIGHT )
( DRONE-MOVE DEPOT ROOM1 DRONE1 )
( DROP-CRATE ROOM1 PERSON1 CRATE1 RIGHT FOOD DRONE1 )

\end{lstlisting}
\begin{lstlisting}[caption={Outcome of the second problem - two people and three boxes}, label={lst:pddl-example}]
( PICK-UP-CRATE CRATE1 DEPOT DRONE1 RIGHT )
( PICK-UP-CRATE CRATE2 DEPOT DRONE1 LEFT )
( DRONE-MOVE DEPOT ROOM1 DRONE1 )
( DROP-CRATE ROOM1 PERSON1 CRATE1 RIGHT FOOD DRONE1 )
( DROP-CRATE ROOM1 PERSON2 CRATE2 LEFT MEDICINE DRONE1 )

\end{lstlisting}

\section{Exercise 1.2: Problem Builder in Python}

\begin{figure}[h]
    \centering
    \includegraphics[width=0.9\textwidth]{Figure_1} 
    \caption{Size of the problem and the time required to find a solution in the tests}
    \label{fig:figure1}
\end{figure}

\section{Exercise 1.3: Performance Comparison of Search Algorithms and
Heuristics}




\end{document}